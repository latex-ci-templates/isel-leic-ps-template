% 'natbib' package
%
% Flexible bibliography support.
% http://www.ctan.org/tex-archive/macros/latex/contrib/natbib/
%
% > produce author-year style citations
%
% \citet  and \citep  for textual and parenthetical citations, respectively
% \citet* and \citep* that print the full author list, and not just the abbreviated one
% \citealt is the same as \citet but without parentheses. Similarly, \citealp is \citep without parentheses
% \citeauthor
% \citeyear
% \citeyearpar
%
%% natbib options can be provided when package is loaded \usepackage[options]{natbib}
%%
%% Following options are valid:
%%
%%   round  -  round parentheses are used (default)
%%   square -  square brackets are used   [option]
%%   curly  -  curly braces are used      {option}
%%   angle  -  angle brackets are used    <option>
%%   semicolon  -  multiple citations separated by semi-colon (default)
%%   colon  - same as semicolon, an earlier confusion
%%   comma  -  separated by comma
%%   authoryear - for author–year citations (default)
%%   numbers-  selects numerical citations
%%   super  -  numerical citations as superscripts, as in Nature
%%   sort   -  sorts multiple citations according to order in ref. list
%%   sort&compress   -  like sort, but also compresses numerical citations
%%   compress - compresses without sorting
%%
% ******************************* SELECT *******************************
%\usepackage{natbib}          % <<<<< References in alphabetical list Correia, Silva, ...
\usepackage[squares, numbers, sort\&compress]{natbib} % <<<<< References in numbered list [1],[2],...
% ******************************* SELECT *******************************


% 'notoccite' package
%
% Prevent trouble from citations in table of contents, etc.
% http://ctan.org/pkg/notoccite
%
% > If you have \cite commands in \section-like commands, or in \caption,
%   the citation will also appear in the table of contents, or list of whatever.
%   If you are also using an unsrt-like bibliography style, these citations will
%   come at the very start of the bibliography, which is confusing. This package
%   suppresses the effect.
%
\usepackage{notoccite}

% ----------------------------------------------------------------------
% Define document language.
% ----------------------------------------------------------------------

% 'inputenc' package
%
% Accept different input encodings.
% http://www.ctan.org/tex-archive/macros/latex/base/
%
% > allows typing non-english text in LaTeX sources.
%
% ******************************* SELECT *******************************
% \usepackage[latin1]{inputenc} % <<<<< Windows
\usepackage[utf8]{inputenc}   % <<<<< Linux
% ******************************* SELECT *******************************


% 'babel' package
%
% Multilingual support for Plain TeX or LaTeX.
% http://www.ctan.org/tex-archive/macros/latex/required/babel/
%
% > sets the variable names according to the language selected
%
% ******************************* SELECT *******************************
% \usepackage[portuguese]{babel}
\usepackage[english]{babel}
% ******************************* SELECT *******************************

% ----------------------------------------------------------------------
% Define default and cover page fonts.
% ----------------------------------------------------------------------

% Use Arial font as default
%
% \renewcommand{\rmdefault}{phv}
% \renewcommand{\sfdefault}{phv}

% Define cover page fonts
%
%         encoding     family       series      shape
%  \usefont{T1}     {phv}=helvetica  {b}=bold    {n}=normal
%                   {ptm}=times      {m}=normal  {sl}=slanted
%                                                {it}=italic
% see more examples at
% https://www.overleaf.com/learn/latex/Font_typefaces
% https://tug.org/FontCatalogue/
%
\newcommand{\FontHn}{% 20 pt normal
    \usefont{T1}{phv}{m}{n}\fontsize{20pt}{20pt}\selectfont}
\newcomand{\FontHb}{% 20 pt bold
    \usefont{T1}{phv}{b}{n}\fontsize{20pt}{20pt}\selectfont}
\newcommand{\FontLn}{% 16 pt normal
    \usefont{T1}{phv}{m}{n}\fontsize{16pt}{16pt}\selectfont}
\newcommand{\FontLb}{% 16 pt bold
    \usefont{T1}{phv}{b}{n}\fontsize{16pt}{16pt}\selectfont}
\newcommand{\FontMn}{% 14 pt normal
    \usefont{T1}{phv}{m}{n}\fontsize{14pt}{14pt}\selectfont}
\newcommand{\FontMb}{% 14 pt bold
    \usefont{T1}{phv}{b}{n}\fontsize{14pt}{14pt}\selectfont}
\newcommand{\FontSn}{% 12 pt normal
    \usefont{T1}{phv}{m}{n}\fontsize{12pt}{12pt}\selectfont}

% 'geometry' package
%
% Flexible and complete interface to document dimensions.
% http://www.ctan.org/tex-archive/macros/latex/contrib/geometry/
%
% > set the page margins (2.5cm minimum in every side)
%
\usepackage{geometry}
\geometry{verbose,tmargin=2.5cm,bmargin=2.5cm,lmargin=2.5cm,rmargin=2.5cm}

% 'setspace' package
%
% Set space between lines.
% http://www.ctan.org/tex-archive/macros/latex/contrib/setspace/
%
% > allow setting line spacing (line spacing of 1.5)
%
\usepackage{setspace}
\renewcommand{\baselinestretch}{1.2}

% 'indentfirst' package
%
% Indent first paragraph after section header.
% https://ctan.org/pkg/indentfirst
%
% > indent all paragraphs (as per IST rules)
%
%\usepackage{indentfirst}

% 'hypcap' package
%
% Adjusting the anchors of captions.
% http://www.ctan.org/tex-archive/macros/latex/contrib/oberdiek/
%
% > fixes the problem with hyperref, that links to floats points
%   below the caption and not at the beginning of the float.
%
\usepackage[figure, table]{hypcap}

% 'booktabs' package
%
% Publication quality tables in LaTeX
% http://www.ctan.org/pkg/booktabs
%
% > enhance the quality of tables in LaTeX, providing extra commands.
%
% \renewcommand{\arraystretch}{<ratio>} % space between rows
%
\usepackage{booktabs}
%\newcommand{\ra}[1]{\renewcommand{\arraystretch}{#1}}

% Typefaces ( example: {\bf Bold text here} )
%
% > pre-defined
%   \bf % bold face
%   \it % italic
%   \tt % typewriter
%
% > newly defined
\newcommand{\tr}[1]{{\ensuremath{\textrm{#1}}}}   % text roman
\newcommand{\tb}[1]{{\ensuremath{\textbf{#1}}}}   % text bold face
\newcommand{\ti}[1]{{\ensuremath{\textit{#1}}}}   % text italic

\usepackage{graphicx}
\usepackage{float}
\usepackage{enumitem}
\usepackage{caption}
\captionsetup[table]{skip=5pt}

\usepackage{titlesec}
%\titlespacing*{<command>}{<left>}{<before-sep>}{<after-sep>}
% left: increases the left margin;
% before-sep: controls the vertical space before the title;
% after-sep: controls the vertical space after the title.
\titlespacing*{\subsection}
{0pt}{1.5ex}{0.5ex}

\titlespacing*{\section}
{0pt}{1.5ex}{1.0ex}

\usepackage{titling}
\setlength{\droptitle}{-60pt} % Adjust the vertical position of the title

\usepackage{hyperref}
\usepackage{hhline}
\usepackage{amstex}

\usepackage[en-US]{datetime2}
